Se deben describir los recursos que se usan en la investigación:

\begin{enumerate}
	\item Maquinaría:
	\begin{itemize}
		\item Nombre.
		\item Describir brevemente el uso adecuado de cada máquina incluyendo, si es necesario, las medidas de seguridad necesarias para su manejo.
		\item Describir el manejo de los deshechos y/o residuos.
	\end{itemize}

	\item Equipos:
	\begin{itemize}
		\item Nombre.
		\item Describir brevemente el uso adecuado de cada equipo incluyendo, si es necesario, las medidas de seguridad necesarias para su manejo.
		\item Describir el manejo de los deshechos y/o residuos.
	\end{itemize}

	\item Químicos de laboratorio:
	\begin{itemize}
		\item Nombre de reactivo.
		\item Indicar los aspectos de buenas prácticas de laboratorio, para el uso del reactivo, con los que se capacitará al equipo de trabajo.
		\item Establecer la disposición de los residuos al término de la experimentación y las medidas de seguridad consideradas para su trazabilidad.
		\item Anexar documentos que autoricen el uso del recurso.
	\end{itemize}
	
	\item Biológicos:
	\begin{itemize}
		\item Nombre del recurso biológico.
		\item Indicar los aspectos de buenas prácticas de laboratorio, para el uso del recurso biológico, con los que se capacitará al equipo de trabajo.
		\item Establecer la disposición de los residuos al término de la experimentación y las medidas de seguridad consideradas para su trazabilidad.
		\item Indicar el origen del recurso biológico. 
		\item Anexar documentos que autoricen el uso del recurso.
	\end{itemize}
	
	\item Renovables:
	\begin{itemize}
		\item Flora
		\begin{itemize}
			\item Nombre de la especie de flora.
			\item Describir las buenas prácticas del manejo del recurso que apliquen a la investigación, considerando manuales, procedimiento y/o normas nacionales o internacionales.
			\item Establecer la disposición de los residuos al término de la experimentación y las medidas de seguridad consideradas para su trazabilidad.
			\item Anexar documentos que autoricen el uso del recurso.
		\end{itemize}
		
		\item Fauna
		\begin{itemize}
			\item Nombre de la especie de fauna.
			\item Describir las buenas prácticas del manejo del recurso que apliquen a la investigación, considerando manuales, procedimiento y/o normas nacionales o internacionales.
			\item Establecer la disposición de los residuos al término de la experimentación y las medidas de seguridad consideradas para su trazabilidad.
			\item Anexar documentos que autoricen el uso del recurso.
		\end{itemize}			
		
		\item Agua
		\begin{itemize}
			\item Indicar la categoría de su clasificación.
			\item Describir brevemente su fuente.
			\item Establecer la disposición de los residuos.
			\item Anexar documentos que autoricen el uso del recurso.
		\end{itemize}
		
		\item Suelos
		\begin{itemize}
			\item Nombre del tipo de suelo.
			\item Descripción breve del tipo de suelo y el origen.
			\item Establecer la disposición de los residuos.
			\item Anexar documentos que autoricen el uso del recurso.
		\end{itemize}
		
		\item Otros
		\begin{itemize}
			\item Nombre de recurso.
			\item Descripción breve del recurso.
			\item Describir brevemente su fuente.
		\end{itemize}
	\end{itemize}
	
	\item No renovables:
	\begin{itemize}
		\item Metálicos
		\begin{itemize}
			\item Nombre de recurso.
			\item Descripción breve del recurso.
			\item Describir brevemente su fuente.
		\end{itemize}
		
		\item No metálicos
		\begin{itemize}
			\item Nombre de recurso.
			\item Descripción breve del recurso.
			\item Describir brevemente su fuente.
		\end{itemize}
		
		\item Combustibles fósiles
		\begin{itemize}
			\item Nombre de recurso.
			\item Descripción breve del recurso.
			\item Describir brevemente su fuente.
		\end{itemize}
		
		\item Radioactivos
		\begin{itemize}
			\item Nombre de recurso.
			\item Descripción breve del recurso.
			\item Describir brevemente su fuente.
		\end{itemize}
		
		\item Otros
		\begin{itemize}
			\item Nombre de recurso.
			\item Descripción breve del recurso.
			\item Describir brevemente su fuente.
		\end{itemize}
	\end{itemize}
	
	\item Materiales nanoestructúrales:
	\begin{itemize}
		\item Nombre.
		\item Descripción breve del material.
		\item Procedimientos de seguridad en su uso y manipulación.
		\item Describir el manejo de los deshechos y/o residuos, si es que lo hay.
		\item Anexar documentos que autoricen el uso del recurso.
	\end{itemize}
	
	\item Información:
	\begin{itemize}
		\item ¿La información implementada ya existe en alguna base de datos? SI/NO
		\begin{itemize}
			\item Descripción breve de qué tipo de información se utilizará para la investigación.
			\item ¿La información pertenece a una institución o empresa del sector privado? SI/NO
			\begin{enumerate}
				\item Describe y adjunta la siguiente documentación: 
				\begin{enumerate}
					\item Formato de solicitud de información.
					\item Permisos de uso de información.
				\end{enumerate}
			\end{enumerate}
		\end{itemize}
		
		\item ¿Su investigación requiere de información obtenida de seres humanos como fuente de información? SI/NO
		\begin{itemize}
			\item Realiza una descripción.
			\item Indica cuales son los criterios de inclusión y exclusión que se implementaron para la selección de los participantes.
			\item Anexa documento de consentimiento informado.
			\item ¿La investigación discrimina la participación de las / los individuos, o incluye un trato diferenciado entre las / los participantes, con base a su género, raza o grupo étnico, edad, religión, ingreso económico, desventaja o discapacidad, enfermedad o cualquier clasificación similar? (Sí es SI, describe el motivo).
			\item ¿La investigación incluye la participación de individuos socialmente o físicamente vulnerables (hombres y mujeres menores de edad, adultos mayores, con capacidades diferentes, etc) o los grupos legalmente restringidos o aislados, o el uso inadecuado de la información puede Confidencial 03/12/2018 Manual de procedimientos DOCUMENTADOS. 4 afectar en algún sentido la integridad de los individuos? (Sí es SI, describe).
		\end{itemize}
	\end{itemize}

\end{enumerate}
